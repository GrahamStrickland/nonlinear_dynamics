% Exercise 2.2.10

\begin{itemize}
    \item[(a)] For the function $\dot{x} = f(x) = 0$, every real number is a fixed
        point, i.e., for $x \in \mathbb{R}, f(x) = 0$.
    \item[(b)] For the function $\dot{x} = f(x) = \sin{\pi x}$, every integer
        is a fixed point, since $\forall x \in \mathbb{R}$, $\sin{\pi x} = 0$.
    \item[(c)] Such a function cannot exist, since $f(x^*) = 0$ implies that either 
        $f(x^* - \varepsilon) < 0$ and $f(x^* + \varepsilon) > 0$ (or vice versa) 
        for some $\varepsilon \in \mathbb{R}$, $\varepsilon > 0$, or 
        $\dot{x} = f(x) = 0$ $\forall x \in \mathbb{R}$, by the intermediate value 
        theorem. Thus there cannot be exactly three stable fixed points of a 
        continuous function.
    \item[(d)] $\dot{x} = f(x) = 1$ has no fixed points, since 
        $\forall x \in \mathbb{R}$, $f(x) \neq 0$.
    \item[(e)] The polynomial
        \[
            \dot{x} = f(x) = (x - \alpha_1)(x - \alpha_2) \cdots (x - \alpha_100),
        \]
        for $\alpha_1 \neq \alpha_2$, $\alpha_1 \neq \alpha_3$, \ldots, 
        $\alpha_2 \neq \alpha_3$, \ldots, $\alpha_{99} \neq \alpha_{100}$ has exactly 
        100 fixed points.
\end{itemize}
