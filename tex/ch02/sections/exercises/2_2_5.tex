% Exercise 2.2.5

In Figure~\ref{fig2_2_5vecfield}, we have a plot of the vector field and various graphs 
with different initial conditions for the equation 
\begin{equation}
    \dot{x} = 1 + \frac{1}{2}\cos{x}.
    \label{eq2_2_5vecfield}
\end{equation}
Note that there are no fixed points for Equation~\eqref{eq2_2_5vecfield}, so that these 
are not included in Figure~\ref{fig2_2_5vecfield}.
\begin{figure}[!ht]
    \includegraphics[scale=0.6, center]{../plots/ch02/fig2_2_5vecfield.pdf}
    \caption{Vector field for $\dot{x} = 1 + \frac{1}{2}\cos{x}$
        \label{fig2_2_5vecfield}}
\end{figure}

We separate equations to find
\begin{equation*}
    \begin{split}
        \frac{dx}{dt} &= 1 + \frac{1}{2}\cos{x} \\
        \Leftrightarrow \int \frac{dx}{1 + \frac{1}{2}\cos{x}} &= \int dt.
    \end{split}
\end{equation*}

We introduce a new variable, $u = \tan{\frac{x}{2}}$, so that we have
\[
    \cos{x} = \frac{1 - u^2}{1 + u^2},
\]
and
\[
    dx = \frac{2}{1 + u^2}dt.
\]

Then we have
\begin{equation*}
    \begin{split}
        \int \frac{dx}{1 + \frac{1}{2}\cos{x}} &= 4\int \frac{du}{3 + u^2} \\
                                               &= \frac{4}{\sqrt{3}}\tan^{-1}
                                               {\biggl(\frac{u}{\sqrt{3}}\biggr)} + c_1 \\
                                               &= \frac{4}{\sqrt{3}}\tan^{-1}
                                               {\biggl(\frac{\tan{\frac{x}{2}}}{\sqrt{3}}\biggr)}
                                               + c_1.
    \end{split}
\end{equation*}

Thus our ordinary differential equation becomes
\[
    \frac{4}{\sqrt{3}}\tan^{-1}{\biggl(\frac{\tan{\frac{x}{2}}}{\sqrt{3}}\biggr)}
    = t + c_2, 
\]
so that we have
\begin{equation*}
    \begin{split}
        \tan^{-1}{\biggl(\frac{\tan{\frac{x}{2}}}{\sqrt{3}}\biggr)}
        &= \frac{\sqrt{3}}{4}(t + c_2) \\
        \Leftrightarrow \frac{\tan{\frac{x}{2}}}{\sqrt{3}}
        &= \tan{\biggl(\frac{\sqrt{3}}{4}t + c_3\biggr)} \\
        \Leftrightarrow \tan{\frac{x}{2}}
        &= \sqrt{3}\tan{\biggl(\frac{\sqrt{3}}{4}t + c_3\biggr)} \\
        \Leftrightarrow x
        &= 2\tan^{-1}\biggl[\sqrt{3}\tan{\biggl(\frac{\sqrt{3}}{4}t + c_3\biggr)} \biggr],
    \end{split}
\end{equation*}
where $c_3 = \frac{\sqrt{3}}{4}c_2$.

Substituting the initial condition $x(0) = 0$ yields
\begin{equation*}
    \begin{split}
        \frac{4}{\sqrt{3}}\tan^{-1}{\biggl(\frac{\tan{0}}{\sqrt{3}}\biggr)}
        &= c_2 \\ 
        \Leftrightarrow \frac{4}{\sqrt{3}}\tan^{-1}{0} &= c_2 \\
        \Leftrightarrow 0 &= c_2,
    \end{split}
\end{equation*}
so that $c_3 = 0$, and we have a solution corresponding to the initial condition $x(0) = 0$
given by
\[
    x(t) = 2\tan^{-1}\biggl[\sqrt{3}\tan{\biggl(\frac{\sqrt{3}}{4}t\biggr)} \biggr].
\]

In Figure~\ref{fig2_2_5graph}, we have a plot of this solution.
\begin{figure}[!ht]
    \includegraphics[scale=0.6, center]{../plots/ch02/fig2_2_5graph.pdf}
    \caption{Graph of a solution of $\dot{x} = 1 + \frac{1}{2}\cos{x}$\label{fig2_2_5graph}}
\end{figure}
