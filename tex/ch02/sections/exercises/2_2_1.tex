% Exercise 2.2.1

In Figure~\ref{fig2_2_1vecfield}, we have a plot of the vector field, fixed points,
and various graphs with different initial conditions for the equation 
\[
    \dot{x} = 4x^2 - 16
\]
\begin{figure}[!ht]
    \includegraphics[scale=0.6, center]{../plots/ch02/fig2_2_1vecfield.pdf}
    \caption{Vector field and fixed points for $\dot{x} = 4x^2 - 16$\label{fig2_2_1vecfield}}
\end{figure}

An explicit solution for $\dot{x} = 4x^2 - 16$ can be found by re-writing the equation as
\[
    \frac{dx}{dt} = 4x^2 - 16,
\]
so that we have separable equations. Then, we may write
\[
    \frac{dx}{4x^2 - 16} = dt,
\]
so that by using partial fraction decomposition, we have
\[
    \biggl[\frac{\frac{1}{16}}{x - 2} - \frac{\frac{1}{16}}{x + 2}\biggr]dx = dt,
\]
and integrate either sides with respect to the $x$ and $t$, respectively, so that we 
have
\begin{equation*}
    \begin{split}
        \int \biggl[\frac{\frac{1}{16}}{x - 2} - \frac{\frac{1}{16}}{x + 2}\biggr]dx
        &= \int dt \\
        \Rightarrow \frac{1}{16}\ln{\lvert x - 2 \rvert} 
        - \frac{1}{16}\ln{\lvert x + 2 \rvert}
        &= t + c_1 \\
        \Rightarrow \ln{\biggl \lvert \frac{x - 2}{x + 2}\biggr \rvert}
        &= 16t + c_2 \\
        \Rightarrow \frac{x - 2}{x + 2} &= \pm e^{16t + c_2}.
    \end{split}
\end{equation*}

After replacing $\pm e^{c_2}$ by $c$ and solving the last equation for $x$, we have
the one-parameter family of solutions
\[
    x = 2\frac{1 + ce^{16t}}{1 - ce^{16t}}.
\]

If we let $x(0) = 0$, we have
\begin{equation*}
    \begin{split}
        2\frac{1 + c}{1 - c} &= 0 \\
        \Rightarrow 1 + c &= 0 \\
        \Rightarrow c &= -1,
    \end{split}
\end{equation*}
so that 
\[
    x_1(t) = 2\frac{1 - e^{16t}}{1 + e^{16t}}
\]
is a solution corresponding to the initial condition $x(0) = 0$.

Likewise, if we let $x(0) = 4$, we have
\begin{equation*}
    \begin{split}
        2\frac{1 + c}{1 - c} &= 4 \\
        \Rightarrow 1 + c &= 2(1 - c) \\
        \Rightarrow c &= \frac{1}{3},
    \end{split}
\end{equation*}
so that 
\[
    x_2(t) = 2\frac{1 + \frac{1}{3}e^{16t}}{1 - \frac{1}{3}e^{16t}}
\]
is a solution corresponding to the initial condition $x(0) = 4$.

Finally, if we let $x(0) = -4$, we have
\begin{equation*}
    \begin{split}
        2\frac{1 + c}{1 - c} &= -4 \\
        \Rightarrow 1 + c &= -2(1 - c) \\
        \Rightarrow c &= 3,
    \end{split}
\end{equation*}
so that 
\[
    x_3(t) = 2\frac{1 + 3e^{16t}}{1 - 3e^{16t}}
\]
is a solution corresponding to the initial condition $x(0) = 4$.

In Figure~\ref{fig2_2_1graph}, we have a plot of these solutions for the given 
initial conditions.
\begin{figure}[!ht]
    \includegraphics[scale=0.6, center]{../plots/ch02/fig2_2_1graph.pdf}
    \caption{Graph of solutions of $\dot{x} = 4x^2 - 16$\label{fig2_2_1graph}}
\end{figure}
